\documentclass[12pt, letterpaper]{article}
\usepackage[margin=1in]{geometry}
\usepackage{amsmath, amssymb, amsthm}
\usepackage{enumitem}
\usepackage{hyperref}
\usepackage{listings}
\usepackage{pythonhighlight}
\usepackage{graphicx}
\usepackage[skip=5pt, indent=20pt]{parskip}

\title{CS 2051: Patterns in Primes}
\author{Anthony Hong \\ Georgia Institute of Technology
\and Yongyu Qiang \\ Georgia Institute of Technology
\and Aravinth Venkatesh Natarajan \\ Georgia Institute of Technology}
\date{March 12th, 2023}

%%%%%%%%%%%%%%%%%%%%%%%%%%%%%%%%%%%%%%%%%%%%%%%%%%%%%%%%%%%%%%%%

\begin{document}

\maketitle

\section{Background}
% This section should develop the basic definitions and
% preliminary results required for the statement and proof of
% your main result, including examples that help the reader to
% develop intuition for the concepts presented. Start from scratch, remember your audience is other 2051 students.
(hook to be added here)

A common result many students learn early on in number theory is about the
infinitude of prime numbers. This fact is also known as Euclid's theorem,
and we include a short summary of Euclid's original proof.

\noindent
\textbf{Euclid's Theorem.} \textit{The set of all prime numbers is larger
in cardinality than any finite collection of prime numbers.}

\noindent
\textit{Proof.} Consider $p_1, p_2, \dots, p_n$, some arbitrary finite
collection of
prime numbers. Let $N = p_1p_2 \dots p_n$, and consider $P = N + 1$. $P$ is
either prime or not prime.
\newline\indent
First, let $P$ be prime. Then, we have constructed a new prime number and
we are done.
\newline\indent
Now, let $P$ not be prime. Let $g$ be a prime factor of $P$. We propose that
$g \notin \{p_1, p_2, \dots, p_n\}$. To show this, suppose that
$g \in \{p_1, p_2, \dots, p_n\}$. Then, since $p_1, p_2, \dots, p_n$ are all
factors of $N$, we have $g | N$. $g | P$ and $g | N$, so
we must also have $g | (P - N)$, i.e.\ $g | 1$. But $g > 1$ ($g$ is prime),
so $g$ cannot possibly divide 1. Therefore,
$g \notin \{p_1, p_2, \dots, p_n\}$, and we have found a new prime, as
required. \hfill$\square$

A natural next step from here is to explore how prime numbers
are distributed. For now, we'll focus particularly on prime
gaps and how small or large they can be. A bit of
thinking leads to the observation that there are certain
restrictions on what prime gaps can look like. First,
we can see that prime gaps can be odd only finitely many
times.

\noindent
\textbf{Lemma.} \textit{There exist only finitely many odd prime
gaps.}

\noindent
\textit{Proof.} Notice that all primes $p > 2$ are odd. Then
$p_{n+1} - p_n$ is even for all $n > 1$, so there exists
only finitely many $n$ such that $p_{n+1} - p_n$ is odd.
\hfill$\square$

In fact, $n = 1$ yields the only odd prime gap, namely
$(p_1, p_2) = (2, 3)$ with difference 1. On the other
hand, we have a much more promising observation for large
prime gaps, namely that we can make them arbitrarily large.

\noindent
\textbf{Lemma.} \textit{There exist prime gaps of arbitrarily
large size.}

\noindent
\textit{Proof.} We'll show that given $n \in \mathbb{Z}^+$, we
can construct an interval of size at least $n - 1$ of only
composite numbers. Then the first primes immediately before
and after this interval will have gap of at least $n$.
\newline\indent
Let $n \in \mathbb{Z}^+$. Now consider the interval
$[n! + 2, n! + n]$. By definition of the
factorial, we have $i|n!$ for all $i \in [2, n]$. We also
trivially have $i|i$. Therefore, we have $i|(n! + i)$, and
so $i$ is a divisor of $n! + i$ for all $i \in [2, n]$.
Then all of $[n! + 2, n! + n]$ is composite, and this
interval has size $n - 1$, so we're done. \hfill$\square$

Although this result is nice, we soon realize that it does
not give us a very strong bound, in the sense that it is
rather wasteful. To find a prime gap of
size $n$ by this method, we must consider numbers of order
$n!$. By Stirling's approximation, we have
\[n! \approx \sqrt{2\pi n} \left(\frac{n}{e}\right)^n\]
asymptotically, which is worse than exponential growth in $n$.
Put in context, our current method suggests that finding
a prime gap of size 10 requires us to find numbers
of magnitude about 3 million. In reality, we can find such a
gap of size 10 at $(p_{30}, p_{31}) = (113, 127)$, which is
much smaller than 3 million, so certainly we can
do better.

For that, we'll need better tools. Let us first define
the prime counting function $\pi(n)$.

\noindent
\textbf{Definition.} $\pi(n) := \text{number of primes} \le n.$

Now we can introduce the prime number theorem,
which characterizes the growth of $\pi(n)$ as $n$ gets large.
Note that we will use this result without proof in this paper,
as even relatively simpler proofs rely on tools from
analysis.

\noindent
\textbf{Prime Number Theorem (PMT).}
$\frac{n}{\ln(n)}$ asymptotically approximates $\pi(n)$.
Put more formally,
$\displaystyle \lim_{n \to \infty} \frac{\pi(n)}{\frac{n}{\ln(n)}} = 1$.

(add more stuff here)

\section{Cram\'er's Random Model}
% In this section, you should state and prove your main result,
% and provide some basic consequences and examples that help the
% reader to understand it. You may want to change this section's
% name to something more informative.
Recall that the PMT tells us that for large $n$,
\[\pi(n) \approx \frac{n}{\ln(n)}.\]
But since we also have $n$ total numbers less than or equal to
$n$, we can divide by this size to get a rough prime density
$\delta(n)$:
\[\delta(n) = \frac{\pi(n)}{n} = \frac{\frac{n}{\ln{n}}}{n} = \frac{1}{\ln(n)}.\]
So for a random number $x$ in the interval $[n, n + kn]$ for
large $n$ and fixed $k$, the probability that $x$ is prime is
approximately $1 / n \approx 1 / x$. (As a side note, this
density $\delta(n)$ is also the rationale behind an alternate
approximation for $\pi(n)$ with the logarithmic integral
$\text{Li}(n)$ as
\[\pi(n) \approx \int_2^{n} \delta(x)\, dx = \int_2^{n} \frac{1}{\ln(x)}\, dx = \text{Li}(n),\]
which actually turns out to be a much better approximation to
$\pi(n)$ than the traditional PMT. In fact, if we assume the
Riemann hypothesis, we have that the error of $\text{Li}(n)$ from
$\pi(n)$ is bounded by $O(n^{1/2 + \epsilon})$ for any
$\epsilon > 0$, meaning that roughly the first half of the
digits of
$\text{Li}(n)$ will be correct,
but that is outside the scope of this paper.)


Cram\'er's random model uses this idea as a naive approach to emulate the distribution of prime numbers. Consider a random subset of the natural numbers, where the independent probability that a number $n$ is chosen is $1 / \ln(n)$. Let's call this random set $P'$, where $P$ is the set of actual prime numbers. Cram\'er conjectured that $P'$, which consists of our ``fake primes," accurately models the distribution of P. 

% Perhaps add summary of proof for this result
According to this heuristic, we have the resulting claim, which is known as Cram\'er's conjecture:
\[\limsup_{n \to \infty} \frac{p_{n + 1} - p_n}{(\ln p_n)^2} = 1\]
where $p_n$ denotes the $n$-th prime.

(Some directions to take these ideas):
\begin{itemize}
  \item Problems with Cram\'er's naive model and ways we can improve it (with modern results)
  \item How Cram\'er's model fares depending on the size and location of the interval, calculating asymptomatic statistics
\end{itemize}

\section{Bertrand's Postulate}
Another problem in number theory is finding the bounds in which you would find a prime number. Of these, one of the paramount significance is \textbf{Bertrand's Postulate}: \\
Bertrand's postulate states that for an integer $i > 1$, there is at least one prime number $p$
\[
    i \leq n \leq 2i 
\]
The proof for it is as follows:\\
We'll start by proving Lemma 1:
\[
  \frac{4^n}{2n} \leq \binom{2n}{n}
\]
\[
    4^n = (1 + 1)^{2n} = \sum{k = 0}^{2n} \binom{2n}{k} 
\]
Since, $\binom{2n}{0}$ is 1 and $\binom{2n}{2n}$ is 1, this is the same as
\[
    \equiv 2 + \sum{k = 1}^{2n - 1} \binom{2n}{k} 
\]
Since the largest term in this summation is $\binom{2n}{n}$ (since for $\binom{n}{k}$, $k = n/2$ will give the largest term) and there are $2n$ terms, 
\[
     (2 + \sum{k = 1}^{2n - 1} \binom{2n}{k}) \leq (2n * \binom{2n}{n})
\]
Therefore, 
\[
    \frac{4^n}{2n} \leq \binom{2n}{n}
\]
Let's now prove Lemma 2:\\
For a given prime $p$, let's define $r$ as the greatest number for which $p^r | \binom{2n}{n}$. Lemma 2 is as follows, for such an r, 
\[
    p^r \le 2n
\]
Firstly, we have to introduce Legendre's Formula:\\
Legendre's Formula states that for any prime number $p$, and any integer $n$, let's define the function $v_p(n)$ as the exponent of the largest power of $p$ that divides $n$. Let L = $\left\lfloor \log_{p}{n} \right\rfloor$\\
Legendre's Formula is:
\[
    v_{p}(n!) = \sum_{i = 1}^{L} {\left\lfloor \frac{n}{p^i} \right\rfloor}
\]
$\binom{2n}{n}$ can also be written as $\frac{(2n)!}{n!*n!})$\\ Finding the largest exponent of $p$, $r$, that divides $\frac{(2n)!}{n!*n!}$ is the same as finding the largest exponent of $p$, $r$, that divides each component of $\frac{(2n)!}{n!*n!}$, i.e. $(2n)!$, $n!$

In this case, $L = \left\lfloor \log_{p}{2n} \right\rfloor$\\ Writing this in terms Legendre's Formula, we get that:
\[
    v_{p}(\binom{2n}{n}) = \sum_{i = 1}^{L} {\left\lfloor \frac{2n}{p^i} \right\rfloor} - 2\sum_{i = 1}^{L} {\left\lfloor \frac{n}{p^i} \right\rfloor}
\]
This is equivalent to:
\[
    v_{p}(\binom{2n}{n}) = \sum_{i = 1}^{L} {\left\lfloor \frac{2n}{p^i} \right\rfloor - 2\left\lfloor \frac{n}{p^i} \right\rfloor}
\]
Thinking intuitively, every term in $\sum_{i = 1}^{L} {\left\lfloor \frac{2n}{p^i} \right\rfloor - 2\left\lfloor \frac{n}{p^i} \right\rfloor}$ must either be 0 or 1.\\
If ($\frac{2n}{p^i}$ mod 1) $\geq$ 0.5 then the term would be 1, otherwise the term is 0. Therefore, the maximum value of this function would be if all the terms were equal to 1. Since we are only dealing with positive numbers, and the exponent is a monotonic function if both numbers are positive numbers,
\[
    v_{p}(\binom{2n}{n}) = r \leq L
\]
\[
    \equiv r \le \log_{p}{2n}
\]
Therefore, it follows that
\[
    p^r \le p^{\log_{p}{2n}} = 2n
\]
We are able to prove our initial Lemma 2, 
\[
    p^r \le 2n
\]
(We will proceed to prove Lemma 3 and 4, which will, in combination prove Bertrand's Postulate)

\section{Extending the Model}
% You should change this section's name to something relevant to
% your project (e.g. "Applications of the RSA encryption
% algorithm" or "Linear Diophantine equations in $n$ unknowns"
\begin{itemize}
    \item Connections from Cram\'er's conjecture to the Riemann hypothesis
    \item Other ways to use Cram\'er's technique of random modeling
\end{itemize}


\section{Preliminary Code and Illustrations}
% Here should be some illustration of the concepts described above. Also place any code you used to generate the code (use the lstlisting package if this applies to you).

Cram\'er's random model allows us to heuristically test properties of primes.
In this example, we graphically compare the maximal prime gap of the model and the actual primes.

\includegraphics[scale=0.7]{graph.png}
\inputpython{example.py}{6}{40}

\section{Accuracy of the Model}
% You should change this section's name to something relevant to
% your project. This section allows you to reflect or conclude
% the work you have done in your project and discuss future work
% that could be done on the project (e.g. applications you tried
% to make but couldn't find the time).

Under heuristic testing with variations of Cram\'er's model, we should be able to support strong statements such as Bertrand's postulate and Legendre's conjecture. However, comparing with the actual primes, it should be clear that Cram\'er's model is inaccurate. Further work would involve the creation and tuning of other random models, to more closely emulate prime distributions.

\section{References}
% Here you should acknowledge people whose help you are thankful
% for (and why), and any sources such as books and websites that
% you used when studying for the project.

\begin{itemize}
    \item \href{http://terrytao.wordpress.com/2015/01/04/254a-supplement-4-probabilistic-models-and-heuristics-for-the-primes-optional/#more-7956}{Terence Tao's Blog}
    \item \href{https://en.wikipedia.org/wiki/Prime_gap}{Wikipedia}
\end{itemize}

 
\end{document}
