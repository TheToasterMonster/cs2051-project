\documentclass{article}
\usepackage[margin=1in]{geometry}
\usepackage{amsmath, amssymb, amsthm}
\usepackage{enumitem}
\usepackage{hyperref}

\title{CS 2051: Project Title}
\author{Anthony Hong \\ Georgia Institute of Technology
\and Yongyu Qiang \\ Georgia Institute of Technology
\and Aravinth Venkatesh Natarajan \\ Georgia Institute of Technology}
\date{}

%%%%%%%%%%%%%%%%%%%%%%%%%%%%%%%%%%%%%%%%%%%%%%%%%%%%%%%%%%%%%%%%

\begin{document}

\maketitle

\section{Background}
% This section should develop the basic definitions and
% preliminary results required for the statement and proof of
% your main result, including examples that help the reader to
% develop intuition for the concepts presented. Start from scratch, remember your audience is other 2051 students.
\begin{itemize}
  \item Definition of $\pi(x)$, the number of primes $\le x$.
  \item Euclid's proof that $\lim_{x \to \infty}\pi(x) = \infty$
  \item Introduce the Riemann zeta function, $\zeta(s) = \sum_{n = 1}^{\infty}\frac{1}{n^s}$
  \item Use the $\zeta(2)$ case and $\frac{\sin{x}}{x}$ Taylor series to show $\zeta(2) = \frac{\pi^2}{6}$
  \item Use above idea to derive Euler product formula
  \item Use Euler product formula with $\zeta(1)$ to get that $\sum_{p}\frac{1}{p} = \infty$, also showing the infinitude of primes
  \item Notation for prime gaps: $g_n = p_{n + 1} - p_n$
  \item A prime gap can be arbitrarily large: Consider $n! + 2, n! + 3, \dots, n! + n$
  \item Are there infinite small prime gaps? Introduce twin prime conjecture
  \item Move on to some other main topic?
\end{itemize}


\section{Main result}
% In this section, you should state and prove your main result,
% and provide some basic consequences and examples that help the
% reader to understand it. You may want to change this section's
% name to something more informative.
A interesting problem is the rate of growth of prime gaps. $\pi(n)$ denotes the prime counting function, which counts the number of primes less than or equal to n. The Prime Number Theorem (PMT) states that $x / \ln(x)$ approximates $\pi(n)$. To be more precise, the PMT states: \[\lim_{x \to \infty} \frac{\pi(x)}{\lfloor \frac{x}{\ln(x)} \rfloor} = 1\] 
This means, as x gets larger, $x / \ln(x)$ will get better as an approximation for $\pi(x)$. This also implies that the average size of the gaps between consecutive primes up until $x$ asymptotically approaches $\ln(x)$. So for a random number $n$ in the interval $[x, x + kx]$ for large $x$ and fixed $k$, the probability that $n$ is prime is approximately $1 / x \approx 1 / n$.

Cram\'er's random model uses this idea as a naive approach to emulate the distribution of prime numbers. Consider a random subset of the natural numbers, where the independent probability that a number $n$ is chosen is $1 / \ln(n)$. Let's call this random set $P'$, where $P$ is the set of actual prime numbers. Cram\'er conjectured that $P'$, which consists of our ``fake primes", accurately models the distribution of P. 

% Perhaps add summary of proof for this result
According to this heuristic, we have the resulting claim, which is known as Cram\'er's conjecture:
\[\limsup_{n \to \infty} \frac{p_{n + 1} - p_n}{(\ln p_n)^2} = 1\]
where $p_n$ denotes the n-th prime.

(Some directions to take these ideas):
\begin{itemize}
  \item Problems with Cram\'er's naive model and ways we can improve it (with modern results)
  \item How Cram\'er's model fares depending on the size and location of the interval, calculating asymptomatic statistics
\end{itemize}
\section{Extension/application/generalisation}
% You should change this section's name to something relevant to
% your project (e.g. "Applications of the RSA encryption
% algorithm" or "Linear Diophantine equations in $n$ unknowns"
\begin{itemize}
    \item Connections from Cram\'er's conjecture to the Riemann hypothesis
    \item Other ways to use Cram\'er's technique of random modeling
\end{itemize}


\section{Preliminary Code and Illustrations}
% Here should be some illustration of the concepts described above. Also place any code you used to generate the code (use the lstlisting package if this applies to you).


\section{Reflection/Conclusion}
% You should change this section's name to something relevant to
% your project. This section allows you to reflect or conclude
% the work you have done in your project and discuss future work
% that could be done on the project (e.g. applications you tried
% to make but couldn't find the time).


\section{References}
% Here you should acknowledge people whose help you are thankful
% for (and why), and any sources such as books and websites that
% you used when studying for the project.

\href{http://terrytao.wordpress.com/2015/01/04/254a-supplement-4-probabilistic-models-and-heuristics-for-the-primes-optional/#more-7956}{Terence Tao's Blog}

 
\end{document}